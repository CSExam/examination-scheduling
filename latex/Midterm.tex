\documentclass[11pt]{beamer}
%\documentclass{article}
\usetheme{Rochester}

\usepackage[T1]{fontenc}
\usepackage[utf8]{inputenc}
%\usepackage[german]{babel}
\usepackage{indentfirst} % \and
\usepackage{hyperref}
\usepackage{multirow}

\usepackage{amsmath}
%\usepackage{amsfonts}
\usepackage{amssymb}
\usepackage{dsfont}
\usepackage{natbib}

\usepackage{graphicx}
\usepackage{tikz}
\usepackage{calc}
\usepackage{ifthen}



\newcommand{\R}{\mathds{R}}     
\newcommand{\N}{\mathds{N}}
\newcommand{\sU}{\mathcal{U}}


\usetikzlibrary{arrows}

%\usepackage{algorithm2e}
%\usepackage{algorithmic}


%% Custom commands
\newcommand{\hide}[1]{}
%\newcommand{\definition}[1]{\textbf{Definition: #1}}
%\newcommand{\algorithm}[1]{\textbf{Algorithm: #1}}
\def\ba#1\ea{\begin{align*}#1\end{align*}}
\def\ban#1\ean{\begin{align}#1\end{align}}
\newenvironment{abschnitt}{}{}

\author{Alexander Eckl, Maximilian Fiedler, Mickael Grima, Roland Halbig}
\title{Examination Scheduling}
% \subtitle{subtitle}
%\logo{}
\institute{Technische Universität München}
\date{March 21, 2016}
%\subject{}
%\setbeamercovered{transparent}
\setbeamertemplate{navigation symbols}{}


\begin{document}
        
        \maketitle
        
        \begin{frame}
               {Konstanten}
               \ba
                    s_i &:= \text{Anzahl der Studenten, die für Prüfung $i$ angemeldet sind} \\
                    c_k &:= \text{Anzahl der nutzbaren Sitzplätzein Raum $k$} \\
                    Q   &:= \text{Kollisionsmatrix} \\
                    q_{i,j} &:= \left\{ \begin{array}{cl} 0, & \mbox{falls Prüfung $i$ und $j$ gleichzeitig stattfinden können} \\ 
                                                          1, & \mbox{sonst} \end{array} \right.  \\
                    T  &:= \text{Sperrmatrix} \\
                    t_{i,j} &:= \left\{ \begin{array}{cl} 1, & \mbox{falls Raum $k$ zum Zeitintervall $l$ geöffnet ist} \\ 
                                                          0, & \mbox{sonst} \end{array} \right.  \\
                    h_l &:= \text{Anzahl der Stunden von Periode $l$ nach Beginn des Prüfungszeitraumes}
               \ea
        \end{frame}
  
  \begin{frame}
               {Variablen}
               \ba
               x_{i,k} &:= \left\{ \begin{array}{cl} 1, & \mbox{wenn Prüfung $i$ im Raum $k$ stattfindet} \\ 
                                                          0, & \mbox{sonst} \end{array} \right.  \\
                y_{i,l} &:= \left\{ \begin{array}{cl} 1, & \mbox{wenn Prüfung $i$ im Zeitinterval $l$ stattfindet} \\ 
                                                        0, & \mbox{sonst} \end{array} \right.  \\
               \ea
               Dimensionen:
              \ba
              n&: Anzahl der Prüfungen \\
              r&: Anzahl der Räume \\
              p&: Anzahl der Zeitintervalle
              \ea
        \end{frame}
       
       
       \begin{frame}
               {Zielfunktion}
               \ba
               \min \sum_{i = 1}^n \sum_{k = 1}^r s_i x_{i,k} - \gamma \sum_{i=1, j>i} \left| \sum_{l=1}^p h_l y_{i,l} - \sum_{m=1}^p h_m y_{j,m} \right|
               \ea
               Resolving the absolute value
               \ba
               \min &\sum_{i = 1}^n \sum_{k = 1}^r s_i x_{i,k} - \gamma \sum_{i=1, j>i} z_{i,j} \\
               s.t. \;\; 
               &z \leq \sum_{l=1}^p h_l y_{i,l} - \sum_{m=1}^p h_m y_{j,m}, \;\; \forall i,j \in [n] \\
               &z \leq -\sum_{l=1}^p h_l y_{i,l} + \sum_{m=1}^p h_m y_{j,m}, \;\; \forall i,j \in [n]
               \ea
        \end{frame}
  
  
       
       \begin{frame}
               {Constraints}
                \begin{enumerate} 
                \item Jede Prüfung wird auf genau einem Zeitinterval eingeplant
                \ba
                    \sum_{l=1}^p y_{i,l} = 1 \;\;\;\forall i \in [n]
                \ea
                \item Alle Studierenden bekommen einen Platz
                \ba
                    \sum_{k=1}^r c_k x_{i,k} \geq s_i \;\;\;\forall i \in [n]
                \ea
               \item Jedem Raum wird je Zeit maximal eine Prüfung zugeteilt
                \ba
                    \sum_{i=1}^n x_{i,k} y_{i,l} \leq t_{k,l} \;\;\;\forall k \in [r], \forall l \in [p]
                \ea
               \item Konfliktvermeidung
                \ba
                    \sum_{i=1, j > i}^n q_{i,j} x_{i,k} x_{j,k} = 0 \;\;\;\forall k \in [r]
                \ea
             \end{enumerate}
        \end{frame}
        
        
        
  \begin{frame}
               {Linear Constraints}
                \begin{enumerate} 
                \item Jede Prüfung wird auf genau einem Zeitinterval eingeplant
                \ba
                    \sum_{l=1}^p y_{i,l} = 1 \;\;\;\forall i \in [n]
                \ea
                \item Alle Studierenden bekommen einen Platz
                \ba
                    \sum_{k=1}^r c_k x_{i,k} \geq s_i \;\;\;\forall i \in [n]
                \ea
               \item Jedem Raum wird je Zeit maximal eine Prüfung zugeteilt
                \ba
                    x_{i,k} + y_{i,l} + x_{j,k} + y_{j,l} \leq 3 , \;\; \forall i,j\in[n], j > i \forall k \forall l \\
                    x_{i,k} + y_{i,l} \leq 1,\;\; \forall i \in [n] \forall k,l \;\;\;\text{such that } t_{k,l} = 0
                \ea
               \item Konfliktvermeidung
                \ba
                    \sum_{j= 1, j \neq i}^n q_{i,j} x_{j,k} \leq (1 - x_{i,k}) ||Q||_2^2 \;\;\;\forall k \in [r] \forall i \in [n]
                \ea
             \end{enumerate}
        \end{frame}
  

\end{document}

